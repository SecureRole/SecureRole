\section{Introduction}
In the digital world, where most of our activities take place online, it is not uncommon to have online cyberattacks. The attackers’ main goal is to steal users’ credentials, which are later used to gain access to accounts, resulting in identity theft and other losses, such as financial losses when credit card information is stolen. \\

The most dangerous, successful and at the same time the simplest attack is phishing, as almost all cyberattacks start with a phishing email. Usually, users are tricked into disclosing their usernames, passwords and other personal information. The attackers forge the email address to make it appear to be from someone else, create fake websites that look like the ones which the user already uses and trusts, and change the URL with foreign character sets. 
When phishing, the attacker may also aim to infect the user’s computer with a malware, by making him download zip files sent as attachments to the email, or even Microsoft Word documents with malicious code embedded. \\

The term phishing is used for a category of attacks consisting of spear phishing, smishing, vishing, and whaling attacks.

\subsection{Spear Phishing}
Spear phishing is when the attacker targets a specific person or organization. The emails targeting the user are specially created for that user with the aim of stealing their credentials and private information.
The attackers try to obtain as much information as possible about the victim beforehand, so that the emails and attachments look more reputable and credible. This way, the chance of deceiving the victim is also higher. The information gathered about the victim is usually accessible online, e.g. on social media. \\
If the attack is successful and the victim provides passwords, PINs, and access codes, the attacker can use this information to steal the victim's identity and access their accounts or even create a new identity with this information.

\subsection{Smishing}
Smishing is a type of phishing that uses short messaging services (SMS) to obtain personal and sensitive information. The SMS is sent to the victim and is worded in such a way that the victim takes immediate action. Example: an SMS is sent to the user that he has won a certain amount of money and that he should enter his personal data via a certain link in order to receive the money, or that a certain account of the user has been compromised and that the account will be lost if he does not act immediately. \\
The attackers can get the phone number from data breaches on the Internet. Or the phone number could be entered directly by the user in a phishing email or on a fake website.


\subsection{Vishing}
Vishing as a term is a combination of the terms \enquote{voice} and \enquote{phishing}. In this case, phone calls are used by the attackers to obtain personal information from the victim. Example: the attacker calls the victim on the weekend or during non-working hours, pretends to be one of the companies whose services the victim uses, and asks the user for more personal and sensitive information.

\subsection{Whaling attack}
The whaling attack is very similar to spear phishing attacks. In the whaling attack, key or important individuals within an organization are targeted. In this case, the phishing emails look much more personalized to the person, making them look more credible. There are also cases where the emails contain company logos and links to non-legitimate websites that look genuine.
Since people in higher positions are targeted here, the attackers invest extra time and effort to make the attack more credible and successful.
Whaling attack is also known as CEO fraud. \\

Spear phishing and whale attacks use emails and non-legitimate websites to obtain personal information from users, while smishing and vishing are attacks that use users' phone numbers to obtain their personal information.