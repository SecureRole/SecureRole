\section{Phishing}

The attackers then launch a spear phishing attack, directly aimed at one of the employees.
They send an e-mail to an accountant containing a link.
The e-mail was specifically intended for him and contained information about one of his clients.
\\

He clicks on the link.
It downloads ransomware directly onto his machine and executes it, directly disabling the security measures from the lsass.exe process.
\\

The following security features have failed to allow this to happen:

\begin{itemize}
    \item SPAM filters are not on high alert, due to the company not experiencing a high SPAM workload.
    \item The employees only received minimal training regarding SPAM awareness.
    \item The company does not employ any DKIM or SPF headers and records.
\end{itemize}

Bellow in \autoref{img:spam_example} 
you will find the e-mail which was sent to the accountant.

\begin{hint}[Important Information]
    You can discuss the SPAM topic with your participants, but this activity should be done \textbf{after} the game, so as not to spoil any information.
    While this decision is up to you, we encourage depriving the students of this additional information for an improved game.
    We will provide additional materials regarding SPAM, so your students can study them on their own or in class.
\end{hint}