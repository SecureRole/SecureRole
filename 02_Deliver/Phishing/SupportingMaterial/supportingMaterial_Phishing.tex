\chapter{Additional material}
\section{Phishing}
The two most important documents created for this game are the game master document, which contains all the essential details for the game master on how to moderate the game, and the character sheets, which are given to each player participating in the game with a description of each role. \\

If you are not that familiar with phishing or want to refresh your memory and repeat some things you learned about phishing a long time ago, our team has created a document explaining phishing as a threat and its mitigations. However, if you are interested more in this topic and need more material, we have listed materials below that might be helpful for you.

\subsection{Publications and research papers}

\begin{itemize}
    \item \href{https://e-tarjome.com/storage/btn_uploaded/2020-09-12/1599891065_11216-etarjome%20English.pdf}{Phishing, Smishing and Vishing}
    \item \href{https://ieeexplore.ieee.org/abstract/document/6497928}{Phishing detection paper}
    \item \href{https://nvlpubs.nist.gov/nistpubs/TechnicalNotes/NIST.TN.1945.pdf}{Email Authentication Mechanisms - SPF, DKIM, DMARC}
\end{itemize}

\subsection{Websites}
For additional information about different categories of phishing and real-world examples, take a look at the following websites:
\begin{itemize}
    \item \href{https://www.pandasecurity.com/en/mediacenter/tips/types-of-phishing/}{11 types of phishing and real life examples}
    \item \href{https://digitalguardian.com/blog/what-is-spear-phishing-defining-and-differentiating-spear-phishing-and-phishing}{Spear phishing; Differences between phishing and spear-phishing}
    \item \href{- https://www.kaspersky.com/resource-center/threats/what-is-smishing-and-how-to-defend-against-it}{What is smishing and how to defend against it}
    \item \href{https://www.trendmicro.com/en_vn/what-is/phishing/social-media-phishing.html}{Social media phishing}
    \item \href{- https://www.barracuda.com/glossary/ceo-fraud}{CEO fraud}
\end{itemize}
On the script created for phishing, there is a small introduction to SPF and how it works. If you want to know more about SPF records and how can they be build we recommend you the following websites:
\begin{itemize}
    \item \href{https://www.dmarcanalyzer.com/spf/how-to-create-an-spf-txt-record/}{How to create an SPF record}
    \item \href{https://www.spf-record.com/syntax}{SPF record syntax}
\end{itemize}

There are also additional mitigations mentioned on the script for phishing, more material on those mitigations can be found on the following websites:
\begin{itemize}
    \item \href{https://www.techtarget.com/searchsecurity/definition/spam-filter}{Spam filter for phishing emails}
    \item \href{https://bitwarden.com/blog/how-password-managers-help-prevent-phishing/}{How password managers prevent phishing}
\end{itemize}


\subsection{Videos}
If you are a visual learner, we have also linked videos to help you better understand phishing in general. \\
To get \textbf{an overview of phishing} and its main categories, we recommend the following video: 
\begin{itemize}
    \item \href{https://www.youtube.com/watch?v=rb26NK0jtHM}{Introduction to phishing and its types}
\end{itemize}

The whole incident in the role-playing game occurs because of a phishing email. If you would like to know some \textbf{quick tips and tricks} on how to detect phishing emails, you can watch the two videos below. Each video contains an example of a phishing email, and it is explained how it can be detected.

\begin{itemize}
    \item \href{https://www.youtube.com/watch?v=j81Wo-khiZA}{How to detect a phishing email - Part 1}
    \item \href{https://www.youtube.com/watch?v=O0Euup8i4kY}{How to detect a phishing email - Part 2}
\end{itemize}

But if you would like to go into \textbf{more details and analyse emails} more, then the following two videos would be suitable for you: 
\begin{itemize}
   \item \href{https://www.youtube.com/watch?v=H4bLUpdFDYo}{Phishing email analysis - Part 1}
   \item \href{https://www.youtube.com/watch?v=-yFkLAveM1Q}{Phishing email analysis - Part 2}
\end{itemize}
There are also helpful videos from the same channel that help spot phishing emails. Again, this can be used as an exercise from your side and check if you would spot the phishing email or not.

\subsection{Audios and Podcasts}
For auditory learners 