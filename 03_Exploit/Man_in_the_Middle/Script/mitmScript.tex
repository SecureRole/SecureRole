
\chapter{Generic Man-in-the-middle attack}
\section{What is a Main-in-the-middle attack?}
A man-in-the-middle attack is a type of attack in which the attacker inserts himself in the "middle" of a communication and eavesdrops the communication between two parties. 
The attacker could also impersonate a legitimate participant in the communication or modify the messages exchanged. \\
Like other types of attacks, also in man-in-the-middle attack the goals are to compromise the: 
\begin{itemize}
    \item \textbf{Confidentiality} of a communication by eavesdropping it 
    \item \textbf{Integrity} of the communication by altering the messages exchanged between two parties
    \item \textbf{Availability} of the communication by attempting to terminate the communication by altering and intercepting the messagesy changing and eavesdroppting the messages
\end{itemize}
There are two main phases which occur during a man-in-the-middle attack: \textbf{interception} and \textbf{decryption}. 

\subsection{Interception}
In the interception phase, the attacker eavesdrops on data transmission between two parties in a communication. 
In this phase, he can also act as a proxy and read, create and update false information into the communication. 
The attacker can trick the user into entering his credentials on an unsecured website, where he steals his credentials and gains access to the account using the stolen credentials.

\subsection{Decryption}
In the decryption phase, all intercepted data is decrypted and deciphered by the attacker.
This allows the attacker to use the data stolen from the user for his own purposes.


\begin{figure}[H]
    \centering
    \includegraphics[width=0.55\textwidth]{./images/mitm.png}
    \caption{Man-in-the-middle attack} 
\end{figure}

\newpage
\section{Main types of Man-in-the-middle attacks}
\subsection{IP spoofing}
IP spoofing is a technique in which IP packets contain a modified source IP address with the aim of disguising the sender of the packet or impersonating another computer system. \\
\\Each IP packet sent contains a header that contains important information for routing, including source and destination addresses. When IP spoofing occurs, the source address of the packet is forged. \\
In this attack, the attacker intercepts communication between two participants to eavesdrop and steal IP packets.
\begin{figure}[H]
    \centering
    \includegraphics[width=0.66\textwidth]{./images/ip-spoofing.png}
    \caption{DNS spoofing attack} 
\end{figure}

\subsubsection{Prevention}
Not much can be done to prevent the attack, but measures can be taken to minimise the damage when that attack occurs.
\begin{itemize}
    \item \textbf{Packet filtering} - checking the source IP address of each packet after it is sent to determine where the packet came from. \\
    The gateway performs ingress filtering, which means that if a packet comes from outside the network but has a source address inside the network, the packet is blocked. \\
    There is also egress packet filtering possible, which controls and verifies the packets sent from the network.
    \item Use IPsec \footnote{IPsec is a secure network protocol suite which authenticates and encrypts packets of data exchanged between two computers over an Internet Protocol network}, so that the risk of IP spoofing is decreased
    \item \textbf{Network Monitoring} - monitor the network for suspicious traffic 
    \item \textbf{Implement firewall} so that suspicious traffic and packets with forged IP addresses can be detected.
\end{itemize}

\newpage

\subsection{Domain Name System (DNS) spoofing}
In this type of attack, domain names are changes and users are redirected to fake websites. 
The fake websites mimic the original ones, and manipulates the user into giving the credentials. 
This way, the attacker gains access to all of the user's credentials. \\

\begin{figure}[H]
    \centering
    \includegraphics[width=0.66\textwidth]{./images/dns-spoofing.png}
    \caption{DNS spoofing attack} 
\end{figure}

When the user sends a DNS request to the DNS server, the attacker injects fake DNS requests into the DNS server and redirects the user to the fake website by sending him fake DNS replies. \\
Since there is no way for the user to know whether the DNS reply is legitimate or not, the user communicates with the malicious website of the attacker.

\subsubsection{Prevention}
\begin{itemize}
    \item Use \textbf{DNSSEC} \footnote{DNSSEC - Domain Name System Security protects data stored in DNS servers} - it helps protect sensitive data stored in DNS server.  It tries to verify the responses sent by DNS servers to the clients, so DNS spoofing can be prevented.
    \item Use only \textbf{secure websites} - double check if the website is legitimate before entering your credentials. HTTPS is an indicator that the website is valid.
    \item \textbf{Patch DNS server} regularly - if DNS servers are not patched, there is a risk thaat the data will become a target of attacks and that it will be exposed. When DNS servers are patched, they remain stable and safe from threats.
    \item \textbf{Monitor DNS traffic} - Watch for new patterns in DNS data or the appearance of a new external host. This could indicate an attacker.
\end{itemize}

\newpage
\subsection{SSL stripping}
As you already know, servers and clients communicate with each other securely using the HTTPS protocol.
This means that they still communicate with each other over HTTP, but there is an additional layer of encryption called SSL on top of it, which enables the encryption and decryption of responses and requests.\\ Transport Layer Protocol (TLS) \footnote{The TLS protocol is a successor of SSL. Today, only TLS is used because SSL versions are considered obsolete and not secure. TLS 1.0 and 1.1 should also be avoided, as they are also outdated.} is an updated version of SSL. Since the abbreviation SSL is better known than TLS, many providers still use the term SSL or the combined term SSL/TLS. \\
\\The main role of SSL is to ensure that only the server can read data sent by the client and that only the client can read data from the server. It also ensures the identity of the server with which the client is communicating by verifying it. \\
Since it is not possible for the attacker to interfere with an HTTPS connection, the attacker tries to downgrade the communication in the unencrypted HTTP protocol. \\
When SSL encryption is removed, all messages including sensitive data such as usernames and passwords are visible and accessible to the attacker. \\

\begin{figure}[H]
    \centering
    \includegraphics[width=0.66\textwidth]{./images/ssl-hijacking.png}
    \caption{SSL Hijacking attack} 
\end{figure}

\subsubsection{Prevention}
\begin{itemize}
    \item Use browser extensions such as \textbf{HTTPS everywhere} - a browser extension designed with the aim of automatically switching http websites to https, making them more secure.
    \item Use \textbf{HTTP Strict Transport Security (HSTS)} - uses a special response header that prevents data from being sent over any communication that takes place over HTTP, and instead sends all communications and data over HTTPS.
\end{itemize}

\newpage
\subsection{Session hijacking}
Sessions are used to store important information for the user and the interaction between two parties participating in the communication. \\
When a user visits a website, the server generates a unique session ID and associates it with the user and that connection. And every time the user makes a new request to the server within the same connection, this session ID is transmitted to the server. The session ID is stored locally in a cookie on the user's computer. \\
When a session is established between the user and the web server, the attacker can steal some parts of the session, such as the browser cookies and gain access to the passwords and other personal information stored in that cookie. \\

\begin{figure}[H]
    \centering
    \includegraphics[width=0.66\textwidth]{./images/session-hijacking.png}
    \caption{Session Hijacking - Attacker sniffing the session} 
\end{figure}


\begin{figure}[H]
    \centering
    \includegraphics[width=0.66\textwidth]{./images/session-hijacking-2.png}
    \caption{Session Hijacking} 
\end{figure}

\subsubsection{Prevention}
\begin{itemize}
    \item Use \textbf{HTTPS} - if SSL/TLS encryption is used during all communication, the attacker will not be able to access and steal the session ID.
    \item Use \textbf{HTTP Strict Transport Security (HSTS)} - as mentioned above, HSTS prevents communication from taking place over HTTP and redirects over HTTPS instead
    \item Use \textbf{HTTPOnly Flag} - to prevent access to the cookies. If the HTTPOnly flag is not set, it means that the cookie can be read by the frontend javascript code. 
    The cookie must be accessible only to the server, so the HTTPOnly flag is used.
    \item \textbf{Regenerate session keys} - generate new session keys after authentication so that the old session key ID cannot be used even if it is stolen by an attacker. \\
\end{itemize}

\subsection{Address Resolution Protocol (ARP) spoofing}
As you may already know, Address Resolution Protocol (ARP) is the protocol that maps an IP address to a MAC address.
A host maintains an ARP cache, which is a mapping table that stores IP addresses and MAC addresses. If the host does not know a particular MAC address, it sends an ARP request packet asking for the MAC address to a particular IP address. \\
The problem, however, is that the ARP protocol cannot verify where the ARP request came from and immediately sends ARP responses. 
This security weakness can be used as an advantage by the attacker.
It associates its address MAC with the two IP addresses of the user and the web server.
Then he interposes himself between the communication, fooling the user and the web server into thinking that they are directly connected, when in fact they are both connected and communicating with the attacker.

\begin{figure}[H]
    \centering
    \includegraphics[width=0.8\textwidth]{./images/arp-spoofing.png}
    \caption{ARP spoofing} 
\end{figure}

\subsubsection{Prevention}
There is no universally applicable solution that can be used to ward off and prevent ARP spoofing.
\begin{itemize}
    \item \textbf{Monitor ARP traffic} in order to detect mapping inconsistencies.
    \item \textbf{MAC Binding} \footnote{MAC Binding means that the MAC address is bound along with the IP address so that the device can no longer access the Internet if either the IP or MAC address changes.} - does not prevent ARP spoofing but prevents MAC spoofing and cloning
\end{itemize}

\newpage
\section{Prevention}
\subsection{Secure connections}
Users should pay attention and visit only secure websites that use only HTTPS and not HTTP.
It is also important to avoid using public WiFi connections because they can be vulnerable to attacks as well, as the attacker can interpose himself between you and the websites you visit. \\
\\In a company, it is important that employees do not use a public network, but an internal network. And this internal network is not used by guests and other external people.

\subsection{Virtual Private Network (VPN) encryption}
VPN encrypts internet connections and data transfers with key-based encryption. Thus, if an attacker accesses the network shared in a VPN, he cannot decrypt the traffic in that VPN.
\subsection{Endpoint security}
\begin{itemize}
\item Ensure endpoints are always using the latest version of a secure browser.
\item Monitor activity on the network to detect abnormal behaviour.
\item Use a password manager for your passwords and do not use the same password twice
\item Keep the system patched
\end{itemize}

\subsection{Multi-factor authentication}
Not all credentials can be stolen from the attacker if multi-factor authentication is implemented. 
This is because the attacker needs not only the username and password, but also another additional form of verification (PIN, or a special code).
In this way, the attacker can be prevented from gaining access to the account even if the credentials are stolen.

\section{Detection}
There are some points that you should pay attention to, that can help detect a Man-in-the-middle attack:
\begin{itemize}
    \item Check the URL of the website. If something seems suspicious, do not visit the website, because it could be DNS hijacking.
    \item Watch out for repeated disconnections from users, because this could be a sign that the attacker is forcing users to disconnect.
    \item Pay attention to the Wi-Fi you connect to.
\end{itemize}

\chapter{Rogue DHCP server}
Dynamic Host Control Protocol is a protocol used for adding and assigning new IP addresses within a local network.
The job of a DHCP server is to assign IP addresses to the hosts.
\begin{itemize}
    \item A \textbf{DHCPDiscover} message is broadcasted from the host to find the DHCP server.
    \item The DHCP Server responds to the host with a \textbf{DHCPOffer} identifying himself as the DHCP server to the host. 
    \item The host sends then a \textbf{DHCPRequest}, asking for an IP address from the server
    \item The DHCP server sends an \textbf{DHCPACK} and acknowledges the IP address assignment to the host
\end{itemize}

\begin{figure}[H]
    \centering
    \includegraphics[width=0.8\textwidth]{./images/dhcp.png}
    \caption{DHCP Communication} 
\end{figure}

\begin{figure}[H]
    \centering
    \includegraphics[width=0.8\textwidth]{./images/dhcp2.png}
    \caption{DHCP - Address assignment} 
\end{figure}

All of these messages are also received by all of the hosts available on the network.
What if one of the hosts on this network is an attacker posing as a legitimate user? What would the scenario look like? \\
When the \textbf{DHCPDiscover} message is sent, the hacker can pretend to be the DHCP server and responds with a \textbf{DHCPOffer} message. If the host receives this message before it receives the \textbf{DHCPOffer} from the real DHCP server, it will believe it. The host will then receive the IP address assigned by the hacker and not by the DHCP server. \\
\begin{figure}[H]
    \centering
    \includegraphics[width=0.8\textwidth]{./images/dhcp-rouge.png}
    \caption{Rogue DHCP server} 
\end{figure}

The attacker's IP address is then assigned as the default gateway for the host. 
This means that when the host wants to connect and communicate with a web server outside the local network, all traffic is first sent to the default gateway, which in this case is assigned to the attacker. \\
After the attacker receives the request and intercepts all the traffic in that request, it forwards the request to the real gateway and then sends it to the server. \\
\begin{figure}[H]
    \centering
    \includegraphics[width=1\textwidth]{./images/dhcp-rouge2.png}
    \caption{Rogue DHCP server} 
\end{figure}

\chapter{Questions}
After reading the script and getting a better understanding of the topic of Man-in-the-middle attack, you can test your knowledge by answering the questions below. \\
All questions can be answered by reading the script, except the last one, which requires a little research. Reading the additional material linked to this topic is not mandatory, but strongly recommended.
\begin{question}[Question]
   Do you know any other types of Man-in-the-Middle attacks? If yes, which? \\
   What are their mitigations?
\end{question}
\begin{question}[Answer]
    \vspace{3cm}
\end{question}

\begin{question}[Question]
    How can Rogue DHCP Server be prevented?
\end{question}
\begin{question}[Answer]
    \vspace{3cm}
\end{question}

\begin{question}[Question]
    What are some mitigations that can be used in Rogue DHCP Server attack?
\end{question}
\begin{question}[Answer]
    \vspace{3cm}
\end{question}
