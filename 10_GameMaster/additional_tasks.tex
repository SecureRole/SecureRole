\section{Additional tasks}

The additional task section contains possible measures the players can take during the game.
They are not part of the storyline, since they could be used at any time.
\\
\begin{hint}[Important Information]
    The goal of this section is to give you, the GM, advice on how to handle players' ideas in line with the game.
    And to give you a helping hand, should the players get stuck.
    You can use these additional tasks to give the participants an idea of how to progress.
\end{hint}
We know that we can't cover every idea a participant could have, so we listed the most common ones.
Should a student have a further idea, you can evaluate it and determine if it could suit the game.

\begin{hint}[Important Information]
    Bellow you can find more detailed information for all additional tasks.
    This should help you as a GM to react accordingly, should your participants choose to consult one of them.
\end{hint}
\subsection{Call the police}

Should the participants choose to contact law enforcement, they will be presented with two options:

\begin{itemize}
    \item Call a police station
    \item Make a report to the NCSC
\end{itemize}

Calling the police will not yield any significant result.
The police will simply help them to file a report in which they detail how the company was attacked and what damage was done, so they can press charges.
\\

They can also go down the route of reporting the incident to the NCSC.
In this case you can show them the pages of the NCSC for reporting an incident.
\\

\begin{itemize}
    \item \href{https://www.ncsc.admin.ch/ncsc/en/home.html}{The homepage of the NCSC}
    \item \href{https://www.report.ncsc.admin.ch/en/}{The reporting page}
    \item \href{https://www.ncsc.admin.ch/ncsc/en/home/infos-fuer/infos-unternehmen/vorfall-was-nun/ransomware.html}{The malware information page}
\end{itemize}

In both cases, they will not receive additional help or support, but simply can file a report.

\begin{hint}[Important Information]
    If you have any digital media at hand, you can show them the webpages.
    Else you can simply tell them they filed a report and the NCSC will look into it as soon as possible.
\end{hint}
\subsection{Create a plan}

Should your participants decide to come up with a plan make sure they do the following:

\begin{itemize}
    \item Set a maximum time budget of 15 minutes.
    \item Include ALL team members (even accounting etc.)
    \item Follow the NIST Incident Response Life Cycle to roughly draw up a plan, as shown in \autoref{img:NIST_principle}
\end{itemize}

\begin{figure}[h!p]
    \centering
    \includegraphics[width=\textwidth]{./images/NIST_principle.png}
    \caption{The NIST process}
    \label{img:NIST_principle}
\end{figure}


\subsection{Get external help}

The participants want to use external help.
It is now your decision whether you want to allow this.
We suggest rejecting this idea (even though it is a great idea that is often followed in practise), but for educational reasons it simply takes away the learning experience when they hire other people to solve their problems.
\\

\begin{hint}[Important Information]
    You can allow for an external entity to fix the issue for them in case they get stuck or if you are short on time and want a satisfactory end to the story.
\end{hint}
\subsection{Media}

\begin{hint}[Important Information]
    It is highly unlikely that your group will make its own media statement.
    So if you want to, you can spice up the game by introducing yourself as journalist who managed to overhear a rumour about their data compromise and demanding answers.
\end{hint}
The group should address what has happened and what they are doing to ensure the protection of their customer data during the breach and in the future.
Make sure they give satisfactory answers to:

\begin{itemize}
    \item How the breach occurred.
    \item What measures are being taken to contain it.
    \item Which future measures are being planned to prevent another occurrence.
    \item If customer data was leaked.
\end{itemize}

\subsection{Shutdown}

The group might come up with the idea of simply shutting down the entire system.
In this case, the spread of the malware will be contained.
\\

It is up to you to judge, depending on the scale of the shutdown, if they can still operate, call their customers and communicate with the hackers.
